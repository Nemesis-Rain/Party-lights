\chapter{Závěr}
Cílem mé dlouhodobé maturitní práce bylo  navrhnout a vyrobit svítící dekoraci pro vnitřní použití 
v počtu
alespoň tří kusů, která bude dálkově ovladatelná, bude umožňovat
měnit barvy světla
a bude pěkně vypadat.

Nejdříve jsem sestavila obvod, který řídí desku ESP32-DevkitC. Tento obvod se zapíná a~vypíná jediným přepínačem, který přerušuje a spojuje obvod. Součástí obvodu je kromě li-ion baterií také regulátor napětí, které napětí baterií sráží na požadovanou hodnotu. Dodatečně jsem k bateriím vytvořila ještě nabíječku, kterou se kdykoliv baterie mohly znovu nabít. 

Navrhla jsem si v aplikaci ovládací panel s tlačítky a naprogramovala ESP32-DevkitC tak, aby se pomocí jeho WiFi modulu daly jednotlivé módy světla z mobilního zařízení bezdrátově ovládat.  

Dále jsem si vytvořila silikonovou formu, do které jsem opakovaně odlila z křišťálové pryskyřice 3 růže, abych je mohla později použít jako žárovky na jednotlivé výrobky. 

Jako poslední jsem navrhla základnu, do které jsem vložila elektrický obvod, k vrchnímu dílu jsem připojila růži, která sloužila jako lampička a skrz kterou LED pásek prosvěcoval do prostoru. 

Vyrobila jsem celkem tři kusy dekorací. Všechna tato zařízení fungují podle očekávání a~splňují parametry zadání. 

Zdrojové soubory a texty k této dlouhodobé maturitní práci jsou umístěny také na mém github účtu \cite{github}.





%Prototyp světla je ke dni odevzdání ročníkové práce plně funkční. Do budoucna by se dal zkonstruovat obal, ve kterém by byla uložená elektronika (hlavně ESP32-DevKitC) a průhledný tvar, skrz který by LED zářily. Dále by se dalo vyrobit a zprovoznit těchto světel víc a zařídit, aby každé z nich bylo propojeno přes wifi s mobilním telefonem.


%Tato ročníková práce mi dala hodně zabrat. Ne proto, že bych měla složité téma a nebo program by byl složitý na naprogramování, ale problém byl v tom, že jsem se musela často učit pracovat s programy a věcmi, se kterými jsem předtím nepracovala. Naučila jsem se pracovat s ESP32 a inteligentními LED světly, naučila jsem se je programovat a dokonce se napojit na ESP32-DevkitC mobilním telefonem a pomocí něj ESP32-DevkitC ovládat.
%Celkově mě ale práce na této ročníkové práci bavila a dala mi do budoucna spoustu zkušeností, které se jistě budou hodit. 

 

